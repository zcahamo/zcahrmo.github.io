\documentclass[a4paper]{article}
\def\coursetitle {W4 Lecture}
\def\coursenum {MATH0057}
\def\nauthor{Aqua Moye}
\input{Coursework_preamble}

% See https://www.overleaf.com/learn/latex/Lists for how to use these
\renewcommand{\labelenumi}{(\alph{enumi})} %\arabic\alph\roman
\renewcommand{\labelenumii}{\alph{enumii}.}

\begin{ques}[6 marks]{1}
	\begin{enumerate}
		\item\label{1a}  A continuous random variable $X$ has cdf $F_X$. Let $Y=aX+b$, for constants $a,b$. If $a > 0$, find an expression for $P(Y \le y)$,
		giving your answer in terms of $a,\ b$ and $F_X$. How, if at all, would this
		expression change if $a < 0$?
		\item  Suppose $X \sim \text{U}(0, 1)$ and that $Y = aX + b$ with $a > 0$. Use the results
		from \hyperref[1a]{part (a)}, together with the expression given in the lecture notes for
		the cdf of a uniform distribution, to find the cdf of $Y$ . Name the corresponding distribution and give the values of its parameters.
	\end{enumerate}
\end{ques}
\begin{enumerate}
	\item For $a>0$, 
	\begin{align*}
		P(Y\le y)&=P(aX+b\le y)=P(X\le \tfrac{y-b}a)=F_X(\tfrac{y-b}a)
		\intertext{For $a<0$,}
	\end{align*}
	
	\begin{align*}
		P(Y\le y)&=P(aX+b\le y)=P(X\ge \tfrac{y-b}a)=1-F_X(\tfrac{y-b}a)
	\end{align*}
	\item \begin{align*}
		F_X(x)&=\begin{cases}
			0&x\le 0\\
		x	&0<x\le 1\\
		1&x>1
		\end{cases}\shortintertext{So}F_Y(y)&=\begin{cases}
		0&\tfrac{y-b}a\le 0\\
		\tfrac{y-b}a	&0<\tfrac{y-b}a\le 1\\
		1&\tfrac{y-b}a>1
		\end{cases}
		=
		\begin{cases}
			0&y\le b\\
			\tfrac{y-b}a	&b<y\le a+b\\
			1&y>a+b
		\end{cases}	
	\end{align*}
	This is a uniform distribution $Y\sim\text{U}(b,a+b)$
\end{enumerate}
\begin{ques}[4 marks]{2}
	For some value $\alpha>1$, a continuous random variable $X$ has pdf
	\[
	f(x)=\begin{cases}\frac{K}{x^\alpha}&x>1\\0&\text{otherwise}\end{cases}
	\]
	where $K$ is an appropriately chosen constant. Find $K$ in terms of $\alpha$. Without
	carrying out any detailed calculations, state the values of $r$ for which the $r^\text{th}$ moment $E(X^r)$ exists. Explain your answer.
\end{ques}

Since any constant is uniquely determined, we may write
\[
f(x)\propto\begin{cases}\frac{1}{x^\alpha}&x>1\\0&\text{otherwise}\end{cases}
\]
Clearly
\begin{align*}
	1&=\int_1^\infty \frac{K}{x^\alpha}\;\d x\\
	&=K\frac{x^{1-\alpha}}{1-\alpha}\bigg|_1^\infty\\
	&=\frac{-K}{1-\alpha}\\
	K&=\alpha-1
\end{align*}
The $r^\text{th}$ moment is
\[
E(X^4)=K\int_1^\infty x^{r-\alpha}\;\d x
\]
This converges for $r-\alpha<-1$, or $r<\alpha-1$.

\begin{ques}[7 marks]{3}
	In a digital camera, battery replacements are assumed to occur in a Poisson process
	of rate $\lambda$ per hour of use. For long-life batteries, $\lambda=0.1$ whereas for normal
	batteries, $\lambda=0.3$. A customer buys a new camera along with a large pack of
	long-life batteries, and is surprised when she has to replace the batteries 3 times
	in the first 10 hours of use. Upon making some enquiries, she discovers that there
	are some counterfeit long-life batteries in circulation, which are normal batteries
	that have been repackaged. It is believed that $5\%$ of long-life battery packets are
	counterfeit. The customer concludes that her batteries are counterfeits.

	\begin{enumerate}
		\item Use an appropriate probability calculation to determine whether or not the
		customer is justified in her conclusion
		\item Do you think the Poisson process model is appropriate in this kind of situation? Justify your answer.
	\end{enumerate}
\end{ques}
\begin{enumerate}
	\item For $C$: the probability the batteries are fake, and $X$: the number of replacements in . Then $X\mid C^c\sim \text{Poi}(1)$,\ $X\mid C\sim \text{Poi}(3)$. Bayes' gives
	\begin{align*}
		P(C\mid X=3)&=\frac{P(X=3\mid C)P(C)}{P(X=3\mid C)P(C)+P(X=3\mid C^c)P(C^c)}\\
		&=\frac{3^3\frac{e^{-3}}{3!}\cdot 0.05}{3^3\frac{e^{-3}}{3!}\cdot 0.05+1^3\frac{e^{-1}}{3!}\cdot 0.95}\approx 0.161
	\end{align*}
	But surely we want to calculate
	\[
	P(C\mid X\ge 3)=\cdots \approx 0.274
	\]
	Hence the customer is justified.
	\item The Poisson model does make sense.
\end{enumerate}
\begin{ques}[3 marks]{4}
	Find the moment generating function of the $\text{U}(a, b)$ distribution.
\end{ques}
\begin{align*}
	M(t)=E(e^{tX})&=\int_\R e^{tX}f(x)\;\d x\\
	&=\int_a^b \frac{e^{tx}}{b-a}\;\d x\\
	&=\frac1{t(b-a)}\left[e^{tx}-e^{tx}\right]_a^b
\end{align*}
\end{document}
