\documentclass[a4paper]{article}
\def\coursetitle {W4 Lecture}
\def\coursenum {MATH0053}
\def\nauthor{Aqua Moye}
\def\centrehead{}
\input{Coursework_preamble}

% See https://www.overleaf.com/learn/latex/Lists for how to use these
\renewcommand{\labelenumi}{\alph{enumi})} %\arabic\alph\roman
\renewcommand{\labelenumii}{\alph{enumii}.}
\iffalse
\begin{ques}[2 marks]{1}
	(2020 exam Q1)
\end{ques}
\fi
\paragraph{Products}
Consider for $A,B\le G$ the product
\[
AB=\{AB:a\in A,\ b\in B\}
\]
\begin{eg}
	Take $C_6=\langle x:x^6=1\rangle$ with $A=\langle x^2\rangle=\{1,x^2,x^4\}$ and $B=\langle x^3\rangle =\{1,x^3\}$. Then
	\[
	AB=\{1,x^3,x^2,x^5,x^4,x\}=C_6
	\]
\end{eg}
\begin{eg}
	Consider $D_8=\langle x,y:x^4=y^2=1,\ yx=x^3y\rangle$. Let $A=\langle y\rangle=\{1,y\}$ and $B=\langle xy\rangle=\{1,xy\}$. Then
	\[
	AB=\{1,xy,y,yxy=x^3\}\not\le D_8
	\]
\end{eg}

In general, a direct product is not a subgroup. However,
\[
A,B\le G\implies A\cap B\le G
\]

For groups $A,B$, the set 
\[
A\times B=\{(a,b):a\in A,b\in B\}
\]
is a group, with operation
\[
(a_1,b_1)(a_2,b_2)=(a_1a_2,b_1b_2)
\]
where elements written next to each other have their implied group operation.

Let
\[
\hat{A}=\{(a,1):a\in A\}\cong A,\quad \hat{B}=\{(1,b):b\in B\}\cong B
\]
See that
\[
(a,1)(1,b)=(a,b)=(1,b)(a,1)
\]
These elements commute. Note that this does mean it is an abelian group, since $A,B$ may not be abelian themselves.

When do we know a group may be written as a direct product? This may be useful, because then the groups $A,B$ may be easier to study.

\paragraph{Quotients}
For a subgroup $H\le G$, consider the left cosets
\[
G/H=\{gH:g\in G\}
\]

\personalcomment{I don't know what we were proving here, something to do with Legrange.}
\begin{proof}[Proof (sketch)]
	For $\sim$ on $G$,
	\[
	x\sim y\iff xH=yH\iff x^{-1}y\in H
	\]
	$\sim$ is an equivalence relation, with equivalence classes of the left cosets. So the the whole group $G$ may be written as a partition into left cosets.
	\[
	G=g_1H\cup g_2H\cup \dots g_m H\implies \abs{G}=m\abs{H}
	\]
\end{proof}
\begin{eg}
	Take $3\Z={0,\pm 3,\dots}\le \Z$. Then
	\[
	\Z/3\Z=\{\bar0,\bar1,\bar2\}
	\]
	Define the group operation on $\Z/3\Z$ of addition modulo 3.
\end{eg}
\begin{prop}
	$G/H$ forms a group if and only if $H\triangleleft\: G$ where $\triangleleft$ has the equivalent definitions
	\begin{gather}
		\tag{0}\forall g\in G\quad gHg^{-1}\subseteq H\\
		\tag{1}\forall g\in G\quad gH=Hg\\
		\tag{2}G/H=H\backslash G\\
		\tag{3}\forall x,y,a,b\in G\qquad x\in aH,\ y\in bH\implies xy\in abH
	\end{gather}
\end{prop}

\begin{eg}Let
	\[
	D_8\langle x,y:x^4=y^2=1,\ xy=x^3y\rangle,\quad A=\langle x\rangle =\{1,x,x^2,x^3\}
	\]
	Then
	\[
	\abs{G}=8,\ \abs{H}=4,\ H\triangleleft G=?,\ G/H=\{H,gH\}=\{H, Hg\}=\{H, G-H\}
	\]
	Subgroup of index 2
	\[
	\abs{D_8/A}=2\implies A\triangleleft D_8
	\]
	Now 
	\[
	B=\langle y\rangle,\ D_8/B=\{B,xB,x^2B,x^3B\},\ xB=\{x,xy\}\ne Bx=\{x,yx\}
	\]
	We know $D_8=AB$, $A\cap B=\{1\},\  A\triangle D_8$. This is just short of the conditions for a direct product, it may be called a semi-direct product.
\end{eg}
\paragraph{Recognition Criterion for Direct Products}
If
\[
\abs{G}=\abs{A}\abs{B},\quad A\triangleleft G,\ B\triangleleft G,\ A\cap B=\{1\}
\]
Then $G\cong A\times B$.

If the orders of $A$ and $B$ are coprime, then their common subgroup can only be $\{1\}$, that is, the condition $A\cap B=\{1\}$ is satisfied.

\begin{eg}
	Reconsider the first example
	\[
	C_6=\langle x:x^6=1\rangle,\quad A=\langle x^2\rangle=\{1,x^2,x^4\},\quad B=\langle x^3\rangle =\{1,x^3\}
	\]
	Then $\abs{A}=2,\ \abs{B}=3,\ A\cap B=\{1\},\ A\triangleleft C_6,\ B\triangleleft C_6$, so
	\[
	C_6\cong C_2\times C_3
	\]
	\personalcomment{Extra comment about commutativity I missed.}
\end{eg}

\paragraph{1st Isomorphism Theorem}
For a group homomorphism $\phi:G\to H$,
\[
G/\ker(\phi)\cong \im(\phi)
\]
This is useful for proving stuff that is obvious but hard to show for some reason.
\begin{eg}
	Try to show $\Z/n\Z\cong C_n$. Take the isomorphism $\phi:\Z\to C_n$ given by $n\mapsto x^k$. This is in fact a homomorphism since
	\[
	\phi(a+b)=x^{a+b}=x^{a}x^{b}=\phi(a)\phi(b)
	\]
	Clearly
	\[
	\im(\phi)=C_n,\quad \ker(\phi)=\{k:x^k=1\}=\{dn:d\in \Z\}=n\Z
	\]
	So what was wanted has been shown.
\end{eg}
\begin{eg}
	Let $\mu:\C^\times\to \C^\times$ given by $\mu(g)=g^2$. This is a homomorphism since
	\[
	\mu(gh)=(gh)^2=g^2h^2=\mu(g)\mu(h)
	\]
	This is subjective since for $z=re^{i\theta},\ r>0$,
	\[
	\mu(\sqrt{r}e^{i\theta/2})=z
	\]
	Also,
	\[
	\ker(\mu)=\{g\in \C^\times:g^2=1\}=\{1,-1\}
	\]
	So
	\[
	\C^\times /\{-1,1\}\cong \C^\times
	\]
\end{eg}

Why is $G/H\cong G$ not true for finite groups with $H\ne \{1\}$? You'd need the same number of left cosets of $G$ as size of $G$ itself. But if $H\ne \{1\}$ then
\[
\abs{G/H}=\frac{\abs{G}}{\abs{H}}<\abs{G}
\]

\paragraph{Homework help}
Take $S_4$ as mentioned previously. The cycles structures, with how many there are, are
\begin{align*}
	[1,1,1,1]&\quad 1\\
	[1,1,2]&\quad \binom{4}{2}=6\quad\text{pick two elements to be in the 2 cycle}\\
	[2,2]&\quad \binom{4}{2}/2=3\quad\text{pick two elements, but half are the same}\\
	[1,3]&\quad \binom{4}{3}\times 2=8\quad\text{pick three elements, then there are 2 orderings}\\
	[4]&\quad \frac{4!}4=6\quad\text{Pick first freely, then 3 choices for 2nd, 2 for 3rd...}
\end{align*}
Check this is right by adding it up to get $4!=24$.
\end{document}
